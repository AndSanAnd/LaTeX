In this work, a new bit selection technique (MTSV), an experimental procedure to evaluate the reliability of SRAM cells for their utilization in power-up PUFs, is exhaustively validated. First, by itself, showing its ability to distinguish unstable, strong and weak cells. Cells selected as strong are consistently shown to be considerably more reliable than those classified as weak under nominal conditions, temperature and supply variations and after accelerated aging. 

Afterwards, MTSV bit selection is tested in the context of using a SRAM PUF as a 128-bit key generator. This is done by comparing the performance of a selection based on MTSV with the performance of other possible selections. When measuring BER under nominal conditions, temperature and supply variations, as well as after accelerated aging, the MTSV-based selections consistently outperform the rest. Next, the ability of these selections to generate a key is tested by implementing a HDA based on the fuzzy commitment construction with a small repetition code. The results from this test demonstrate how MTSV-based selections are the only ones able to always recover the key successfully under different operating conditions. Finally, the required repetition code to achieve a KER lower than $10^{-4} \%$ for each selection is calculated. Although further measurements are still necessary to completely evaluate this method, specially at high temperatures, the results show quantitatively how much the requirements for the ECC are reduced, with a selection based on MTSV needing three to five times less redundant bits than the other selections. This considerably reduces the cost of implementation of SRAM PUFs, a matter of great importance due to their use in resource-constrained devices. The MTSV bit-selection technique represents one more step in the development of SRAM PUFs.

Future work will focus on implementing the MTSV technique in a commercial chip, including the enrollment and reconstruction phases, to demonstrate its capabilities in real implementations. Another step further would be to perform the hardware design to implement the complete crytographic solution on an IC, as well as evaluating the aging resilience of this design.
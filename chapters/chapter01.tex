
Every year, connectivity and digitization increases in almost every aspect of our lives. This growing tendency has transformed many fields, such as communications, healthcare, finance or transportation, providing new solutions and a massive increase in efficiency and productivity. The deployment of IoT networks, where multiple devices– from simple sensors to smartphones and wearables – are connected together for the purpose of exchanging data over the network, have been accelerated during the last years. However, there is a lack of security in digital interconnectivity, thus making the exploitation of sensitive data through device impersonation very profitable for potential attackers. 

A way to protect a digital device is the integration of a hardware Root-of-Trust (RoT). Conventional implementations of hardware RoT employ Non-Volatile Memories (NVMs) to store secret keys. This is an expensive solution due to their high design area and power consumption. Moreover, NVMs are vulnerable to physical attacks being necessary to add active circuitry that detects tampering, increasing the cost even more. 

Furthermore, several IoT scenarios often operate on resource-constrained devices, which makes the use of NVMs non-affordable. Physical Unclonable Functions (PUFs) have emerged as an alternative to solve this problem, thus being one of the dominant topics in the hardware security domain. A PUF is a physical implementation of a function that maps an input (challenge) to an output (response). Specifically, silicon PUFs are based on exploiting small variations present during semiconductor manufacturing, so that the PUF creates a unique challenge-response mapping and thus it cannot be cloned. PUFs can be used as the foundation of a large set of security related tasks such as authentication and generation of cryptographic keys. These features make PUFs key elements to build RoT. PUFs usually imply simple circuitry and they generally do not need anti-tamper protection, since physical attacks on PUFs are very difficult to perform without modifying the characteristics from which the RoT is derived. 

There are multiple implementations for silicon PUFs, but SRAM-based PUFs are one the most popular ones due to their low-cost of implementation \cite{McGrath2019,Bohm2013}. For this reason, this type of PUFs will be addressed in this work. In SRAM PUFs, the power-up values of the SRAM memory cells are used as unique identifiers. Commercial implementations of SRAM PUFs are already available through hardware security solution suppliers such as NXP \cite{NXP}, intrinsic ID \cite{Intrinsic} and Microsemi \cite{Microsemi}.

The main drawback of SRAM PUFs is their limited reliability, as they do not always return the same response to a challenge, an essential feature of a PUF. Reliability is further diminished by environmental conditions during operation and device aging. This problem can be solved by utilizing a series of pre- and post-processing techniques, which, together, form the helper data algorithm (HDA) \cite{Delvaux2015}. Different techniques are available, and the development of new ones has been an important topic of research for the past years \cite{Hiller2020,Alioto2019,Shifman2018}. Particularly, a common post-processing technique is the use of error correction code (ECC) circuits, to ensure a correct response. However, this circuitry implies a high cost in terms of area, power and the requirement of redundant bits out of the PUF, a cost that increases linearly with the unreliability of the PUF. As one of the main advantages of PUFs is their low cost, there is a great incentive to increase reliability as much as possible with the goal of reducing the complexity of ECC. 

One way to increase reliability in SRAM-PUFs before applying the ECC is through bit selection, which reduces the incorrect responses by selecting the most reliable cells of the array. The main goal of this project is to exhaustively validate through experimental results a new bit selection technique. It performs a better selection than previously reported techniques, making possible a significant reduction of the ECC circuitry needed. The method is experimentally validated against voltage and temperature variations and aging effects in an SRAM array designed in a 65nm CMOS technology. 

In chapter \ref{chap:2} an introduction to PUFs is presented, including some of the available post-processing techniques to improve their reliability. Then, a more detailed look into SRAM PUFs is done in chapter \ref{chap:3}, including the effects of environmental variations and aging mechanisms on reliability. Afterwards, in chapter \ref{chap:4}, a variety of SRAM-specific pre-processing techniques are explained, which serve as point of reference for the new pre-processing technique, Maximum Trip Supply Voltage (MTSV). This technique is validated at different operating conditions, demonstrating its benefits even in the case of aging of the SRAM cells. Finally, in chapter \ref{chap:5}, the process by which an SRAM PUF's unreliable response is transformed into a reliable one is evaluated in detail using a fully characterized SRAM PUF chip. Through this process, the performance of a selection of cells based on MTSV is compared to other selections, which will illustrate the degree of improvement achieved thanks to MTSV. Afterwards, the ability of these selections to generate a key with an ECC is tested. Finally, the reduction in cost achieved by MTSV is shown by comparing the required ECC for a selection based on MTSV and the requirements for other selections. 



% Taken from A Novel SRAM PUF Stability Improvement Method
% Using Ionization Irradiation
% Unlike non-silicon PUFs, SRAM power-on values can be directly read as binary codes, so there
% is no need for complex PUF response extraction operations \cite{McGrath2019}. Compared to RO PUF and Arbiter PUF, SRAM PUF can reuse on-chip SRAM memory or commercial off-the-shelf SRAM chips \cite{Zhang2020}. This dramatically decreases the product development costs and eases integration difficulties. Due to the above factors, SRAM PUF increasingly attracts researchers’ exploration around the world, it is also the most popular commercial PUF component which is valued by advanced hardware security solution suppliers such as NXP \cite{NXP}, intrinsic ID \cite{Intrinsic} and Microsemi \cite{Microsemi} and so on.